\subsection{GsModel .m Model Definition File}

{\footnotesize  (Status: needs revision with respect to the new GSModel internal format including grouped information.}

\begin{lstlisting}[title={}]
GsModel           # signature to identify the file

[name <name> ]    # if not given, name becomes an empty string

[culling <0|1>]   # if not defined, back-face culling is on by default

vertices <nv>     # list of vertices is mandatory, nv is the number of vertices
<x> <y> <z>
...
  
faces <nf>        # list of triangular faces is mandatory
<a> <b> <c>       # a triangle is defined with indices to the vertex list, starting from 0
...

[normals <nm>     # optional list of normals per vertex
<x> <y> <z>
...]

[fnormals <nf>    # optional list of normals per face
<a> <b> <c> 
...]

[materials <nm>   # list of materials, each material is read by GsMaterial input operator
amb <r> <g> <b> <a> dif <r> <g> <b> <a> spe <r> <g> <b> <a> emi <r> <g> <b> <a> shi <v> [tid <i>]
...]

[fmaterials <mf>  # indices of materials to be assigned per face
<i1>
<i2>
...]

[mtlnames         # optional names for each material index
i name1
i name2
...]

[textcoords <nt>  # texture coordinates defined per vertex
<u> <v>
...]

[textures <nt>
<image.png>       # image file for each texture to be used
...]

[ftextcoords <nf> # texture coordinates as indices per face
<a> <b> <c>
...
<a> <b> <c>]

[primitive        # if the model represents a primitive, the primitive parameters go here
 <box|sphere|cylinder|capsule> <ra> [rb] [rc] <nfaces> #(nfaces needed even for a box)
 [center <x y z>]
 [orientation axis <x y z> ang <deg>]
 [material amb <r> <g> <b> <a> dif <r> <g> <b> <a> spe <r> <g> <b> <a> emi <r> <g> <b> <a> shi <v>]
 [color <r> <g> <b> <a>] # color will set the diffuse color of the default material
 [smooth|flat]
 ;
]

\end{lstlisting}