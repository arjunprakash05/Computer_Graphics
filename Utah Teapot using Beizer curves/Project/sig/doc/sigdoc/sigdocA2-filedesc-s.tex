\subsection{KnSkeleton .s Skeleton Definition File}

\begin{lstlisting}[title={}]
KnSkeleton                   # Signature. It should be the first keyword of the file.

[ path|add_path <path> ]     # Loaded geometry files will be searched in the 
                             # same directory of the .s file, or also searched
                             # in the directories specified with the path command.
                             # Several paths can be defined to be searched.

[ name|set_name <skelname> ] # Specifies the name of the skeleton


[ scale <scale factor> ]     # Command scale is used to scale the length of the
                             # skeleton links (offsets), for example for converting
                             # units. Translational limits are also scaled.

[ globalgeo <true|false> ]   # If true, the geometry is considered to be loaded in
                             # global coordinates and therefore all geometries are
                             # converted to local coordinates after loaded.
                             # By default globalgeo is considered false.

<SKELETON_DEFINITION>        # Usually the skeleton definition comes here. The 
                             # definition syntax is specified later on in this file.

[ posture <name val1 val2 ... valN> ] # Command posture specifies a posture which
                             # values must match the active channels of the
                             # skeleton. All loaded postures will share the
                             # same channels. This command must come after the 
                             # skeleton definition.

[ dist_func_joints <joint1 joint2 ... jointN;>] # This command specifies which 
                             # joints are used in the distance function between
                             # postures. This command must come after all posture
                             # definitions.

[ collision_free_pairs  <jointname1 jointname2 ...>; ] # List of joint pairs to
                             # to be deactivated for collision detection.

[ userdata <var1=val; var2=val1 val2; ... varN=val;> ] # This command allows the
                             # specification of any kind of user-related data.
                             # The data is loaded as a GsVars object that is
                             # maintained by the skeleton

[ ik <LeftArm|RightArm|LeftLeg|RightLeg> <jointname> ] # initialize IK with given
                                                       # joint as end effector

[ end ]                      # Optional keywork that forces end of parsing


#### SKELETON_DEFINITION ####
# A skeleton can be defined in two ways: hierarchical or flat.
# The syntax of a hierarchical skeleton definition is:

skeleton                     # tells this is a hierarchical definition
root <name>                  # specifies the root joint and its name
{ <JOINTDEFS>                # joint definitions go here
  joint <name>               # specifies child joint and name
   { <JOINTDEFS>             # etc
     joint <name>
      { ...
      }
   }
  ...
}

# The syntax of a flat non-hierarchical skeleton definition is:

root <name>                  # first specify the root joint
{ <JOINTDEFS>
}
joint <name> : <parent name> # then specify each joint with its name and its
{ <JOINTDEFS>                # parent's name, which must be previously defined
}
...                          # etc

# It is possible to modify the settings of a previously defined joint with
# the following syntax (the syntax of a .sd file):
joint <name>
{ (JOINTDEFS)
}

#### JOINTDEFS ####
# The possible joint definitions are listed below.
# Rotations specified in "axis <x y z> ang <a>" can also be written as "x y z a",
# where a is an angle in degrees.

  [offset|center <x y z> ]

  [euler XYZ|YXZ|YZX|ZY]

  [channel XPos|YPos|ZPos|XRot|YRot|ZRot <val> [free | <min><max> | lim <min><max>] ]

  [channel Quat [axis <x> <y> <z> ang <degrees>] [frozen] ]

  [channel Swing [axis <x> <y> ang <degrees>] [lim <xradius> <yradius>] ]

  [channel Twist <val> [free | <min><max> | lim <min><max>] ]

  [modelmat <4x4matrix as 16 floats>]          # apply to the joint model

  [modelrot <axis <x> <y> <z> ang <degrees>>]  # apply to the joint model

  [prerot <axis <x> <y> <z> ang <degrees>>]    # joint pre rotation

  [postrot <axis <x> <y> <z> ang <degrees>>]   # joint post rotation

  [align <pre|post|prepost|preinv> <x y z>]    # set pre/post for aligning given vector

  [visgeo <model filename>] # models are "added" if more than one visgeo is declared

  [colgeo <model filename>] # models are "added" if more than one colgeo is declared

  [visgeo|colgeo primitive <defs>] # see Note 2 below
  
  [visgeo|colgeo shared]    # reuse the other col/visgeo previously defined in the joint


#### Advanced Notes ####
- Note1: Quaternion rotations can now be also loaded with format: xzy <x> <y> <z>
- Note2: Geometries can now also be created with [visgeo|colgeo primitive <defs>;],
         where <defs> are the commands in the primitive description of the .m format

\end{lstlisting}
